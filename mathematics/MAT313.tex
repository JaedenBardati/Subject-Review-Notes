\documentclass[]{article}


\usepackage{amsthm}
\newtheorem{definition}{Definition}	% definition


\setlength{\parindent}{0pt}		% no indent

%opening
\title{
	MAT 313 \\
	\large Introduction to Probability
}
\author{Jaeden Bardati}
\date{}

\begin{document}

\maketitle

\section{Introduction}

Probability theory is a tool to model and study \textbf{uncertainty}.\\

We consider collected data as a result of measuring some numerical attributes of outcomes of random experiments.\\

Many scientific experiments are similar to games of chance in the sense that multiple trials of the same procedure can lead ot many different results (varying from one trial to the next).\\

A realistic model of real-world phenomenon must take into account the possibility of randomness.\\

Often the quantities we are interested in will not be predictable in advance; they will exhibit an inherent variation that should be taken in to account by. \\

\begin{definition}
	The set of all possible outcomes of an experiment is called the \textbf{sample space} of the experiment and is denoted by the symbol $S$.
	An outcome in a sample space is called an \textbf{element} or a \textbf{member} of the sample space.
\end{definition}





\end{document}
