\documentclass[]{article}

% --- PACKAGES ---

\usepackage{titling}	% for subtitle custom command
\usepackage{amsmath}	% for many math things like \begin{align}
\usepackage[thinc]{esdiff}	% for derivatives and partial derivatives

% Fixes weird backwards quote thing
\usepackage [english]{babel}
\usepackage [autostyle, english = american]{csquotes}
\MakeOuterQuote{"}

% ----------------

% --- CUSTOM COMMANDS ---
\newcommand{\subtitle}[1]{
	\posttitle{
		\par\end{center}
	\begin{center}\large#1\end{center}
	\vskip0.5em}
}

% -----------------------

% stop indentation
\setlength{\parindent}{0pt}

% Title
\title{PHY 474 Notes}
\subtitle{Cosmology}
\author{Jaeden Bardati}

\begin{document}

\maketitle
\bigbreak

% Course Video 1
\section{Introduction to Cosmology}
\bigbreak

\subsection{Isotropic and Homogeneous Space-times\\ {\large \normalfont Sept. 9, 2021}}
\bigbreak

The universe is \underline{homogenous} and \underline{isotropic}.\\

\begin{itemize}
	\renewcommand{\labelitemi}{$\Rightarrow$}
	\item \textbf{Homogenous}: Looks the same everywhere
	\item \textbf{Isotropic}: Looks the same in all directions\\
\end{itemize}

Observational evidence that support this assumption are from the Cosmic Microwave Background (CMB) and large-scale structure observations. \\

\subsection{Expansion of the Universe\\ {\large \normalfont Sept. 9, 2021}}
\bigbreak

\textbf{Redshift} ($z$): The doppler shift of the light from galaxies observed through emission/absorption lines in spectra.\\

\begin{equation}
	z = \frac{\lambda_{\mathrm{observed}} - \lambda_{\mathrm{emitted}}}{\lambda_{\mathrm{emitted}}}
\end{equation}\\

For an expanding universe, $z_{\mathrm{galaxy}} > 0$.\\\\
The larger $z$ is, the further the distance the galaxy is. 

\subsection{Hubble Constant\\ {\large \normalfont Sept. 9, 2021}}
\bigbreak

Imagine 3 galaxies. Let the relative distances, at  t = $t_0$, be $r_{12}(t_0)$, $r_{23}(t_0)$, and $r_{13}(t_0)$. In an expanding universe, the distances increase by a scale factor $a(t)$. Thus,

\begin{align*}
	r_{12}(t) &= a(t)~r_{12}(t_0)\\
	r_{23}(t) &= a(t)~r_{23}(t_0)\\
	r_{13}(t) &= a(t)~r_{13}(t_0)
\end{align*}\\

Imagine you are an observer in galaxy 1. You will see galaxies 2 and 3 moving with velocity:

\begin{align*}
	v_{12} &= \diff{r_{12}}{t} \\
	&= \dot{a}~r_{12}(t_0)\\
	&= \frac{\dot{a}}{a}~r_{12}(t)
\end{align*}\\

A similar relation is true for $v_{23}$ and $v_{13}$.\\\\

The factor $\frac{\dot{a}}{a}$ relates the velocity of galaxies to their distance away from us. It is called the Hubble constant ($H_0 \equiv \frac{\dot{a}}{a}$). Thus for galaxies:

\begin{equation}
	v = H_0 r
\end{equation}\\

This relation can be described with the Hubble diagram. It is a graph of velocity vs. distance of a bunch of galaxies. There is a linear trend with slop of $H_0$.\\\\

Doppler shift is $z = \frac{v}{c}$, so

\begin{equation}
	z = \frac{H_0}{c}r
\end{equation}\\

\subsection{Age of the Universe\\ {\large \normalfont Sept. 9, 2021}}
\bigbreak

$H_0$ is measured to be around 70 km/s /Mpc. If we assume a constant $H_0$, we can calculate an age of the universe:

\begin{equation*}
	t_0 = \frac{r}{v} = \frac{r}{H_0 r} = \frac{1}{H_0}
\end{equation*}\\

For $H_0 \approx$ 70 km/s/Mpc, $t_0 \approx$ 14 Gyr.\\


\subsection{Contents of the Universe\\ {\large \normalfont Sept. 9, 2021}}
\bigbreak

\begin{center}
	\begin{tabular}{lcr}
		Dark matter & & 25\% \\
		Dark energy & & ~75\% \\
		Stars & & 0.2\% \\
		All baryons & & 5\% \\
		Radiation (primarily CMB) & & 0.01\% \\
	\end{tabular}
\end{center}\bigbreak

\subsection{Timeline of the Universe\\ {\large \normalfont Sept. 9, 2021}}
\bigbreak

\begin{center}
	\begin{tabular}{lcl}
		$t = 0$ & & Big Bang \\
		$t = 10^{-39}$ s & & Classical gravity breaks down \\
		$t = 10^{-35}$ s & & Inflation \\ 
						 & & \textit{(density fluctuations formed)} \\
		$t = 10^{-14}$ s & & Baryogenesis  \\
						 & & \textit{(formation of protons/electrons and} \\
						 & & ~\textit{matter dominates over anti-matter)}\\
		$t = 100$ s & & Big Bang nucleosynthesis \\
					& & \textit{(formation of Hydrogen nuclei)} \\
		$t = 5 \times 10^{4}$ s & & Matter dominates over radiation \\
		$t = 4 \times 10^{5}$ yr & & Recombination \\
								 & & \textit{(formation of neutron atoms and} \\
								 & & ~\textit{the CMB was emitted)} \\
		$t = 100$ Myr & & First population III stars form \\
		$t = 500$ Myr & & First galaxies form \\
		$t = 0.5$-1 Gyr & & Reionization or "Cosmic Dawn" \\
		$t = 2$-3 Gyr & & "Cosmic Noon" \\ 
					  & & \textit{(peak of stellar formation and SMBH accretion)} \\
		$t = 10$ Gyr & & Dark energy dominates \\
		$t = 14$ Gyr & & Present day \\
	\end{tabular}
\end{center}\bigbreak

\subsection{Geometries of isotropic and homogeneous space-times\\ {\large \normalfont Sept. 13, 2021}}
\bigbreak

In cosmology, we often need to measure distances in an expanding (and sometimes curved) space-time.\\

Space can be flat(Euclidean) with ($k = 0$), positively curved ($k = +1$), or negatively curved ($k = -1$).

\subsubsection{In 2D}
\bigbreak

In flat space, 




\end{document}
